% Time-stamp: <Sunday 16 April 2023, Romain Lafarguette>
% Romain Lafarguette 2020, https://romainlafarguette.github.io/

%% ---------------------------------------------------------------------------
%% Preamble: Packages and Setup
%% ---------------------------------------------------------------------------
% Class 
\documentclass{beamer}

% Theme
\usetheme{Boadilla}
\usecolortheme{dolphin}
%\setbeamertemplate{headline}{} % Remove the top navigation bar

% Font and encoding
\usepackage[utf8]{inputenc} % Input font
\usepackage[T1]{fontenc} % Output font
\usepackage{lmodern} % Standard LateX font
\usefonttheme{serif} % Standard LateX font

% Maths 
\usepackage{amsfonts, amsmath, mathabx, bm, bbm} % Maths Fonts

% Graphics
\usepackage{graphicx} % Insert graphics
\usepackage{subfig} % Multiple figures in one graphic
\graphicspath{{/img}}

% Layout
\usepackage{changepage}

% Colors
\usepackage{xcolor}
\definecolor{imfblue}{RGB}{0,76,151} % Official IMF color
\setbeamercolor{title}{fg=imfblue}
\setbeamercolor{frametitle}{fg=imfblue}
\setbeamercolor{structure}{fg=imfblue}

% Tables
\usepackage{booktabs,rotating,multirow} % Tabular rules and other macros
%\usepackage{pdflscape,afterpage} % Landscape mode and afterpage
%\usepackage{threeparttable} % Split long tables
\usepackage[font=scriptsize,labelfont=scriptsize,labelfont={color=imfblue}]{caption}

% Import files
\usepackage{import}

% Appendix slides
\usepackage{appendixnumberbeamer} % Manage page numbers for appendix slides

% References
\usepackage{hyperref}

% A few macros: environments
\newenvironment{wideitemize}{\itemize\addtolength{\itemsep}{10pt}}{\enditemize}
\newenvironment{wideenumerate}{\enumerate\addtolength{\itemsep}{10pt}}{\endenumerate}

\newenvironment{extrawideitemize}{\itemize\addtolength{\itemsep}{30pt}}{\enditemize}
\newenvironment{extrawideenumerate}{\enumerate\addtolength{\itemsep}{30pt}}{\endenumerate}

% Remove navigation symbols and other superfluous elements
\setbeamertemplate{navigation symbols}{}
\beamertemplatenavigationsymbolsempty

%\setbeamertemplate{note page}[plain]
\hypersetup{pdfpagemode=UseNone} % don't show bookmarks on initial view
\setbeameroption{hide notes}

% Institute font
\setbeamerfont{institute}{size=\footnotesize}
\DeclareMathSizes{10}{9}{7}{5}

% Footnote without marker
\newcommand\blfootnote[1]{%
  \begingroup
  \renewcommand\thefootnote{}\footnote{#1}%
  \addtocounter{footnote}{-1}%
  \endgroup
}

%% ---------------------------------------------------------------------------
%% Title info
%% ---------------------------------------------------------------------------
\title[Introduction]{STI Course on FX Interventions \\ General Introduction}

\author[Lafarguette, Raboun, Chen]{Romain Lafarguette, Ph.D. \and Amine Raboun, Ph.D. \\ \and Zhuohui Chen}
\institute[IMF]{ADIA Quants \& IMF External Experts \& IMF MCM \blfootnote{\scriptsize{\emph{This training material is the property of the IMF, any reuse requires IMF permission}}} \\
\begin{center}{\href{https://romainlafarguette.github.io/}{\textcolor{imfblue}{www.romainlafarguette.github.io}} \hspace{0.3cm} \href{https://amineraboun.github.io/}{\textcolor{imfblue}{www.amineraboun.github.io}}} \end{center} \vspace{-0.5cm}} 

\date[STI, 17 April 2023]{\footnotesize Singapore Training Institute, 17 April 2023}

\titlegraphic{\vspace{-0.6cm}
    \begin{figure}
    \centering
    \subfloat{{\includegraphics[width=2cm]{img/imf_logo}}}%
    \end{figure}}

  
%% ---------------------------------------------------------------------------
%% Title slide
%% ---------------------------------------------------------------------------
\begin{document}

\begin{frame}
\maketitle
\end{frame}


%% ---------------------------------------------------------------------------
%% Main Body
%% ---------------------------------------------------------------------------
\begin{frame}
  \frametitle{The Course Website}

Available via: \href{https://amineraboun.github.io/STI_FX_Intervention/docs/index.html}{\beamergotobutton{Course Website}}

\makebox[\linewidth]{\includegraphics[height=0.75\paperheight]{img/website.PNG}}
    
\end{frame}


\begin{frame}
  \frametitle{Overview of the Course}
  \begin{wideenumerate}
  \item Conceptual framework and academic literature on FX interventions 
    \begin{itemize}
    \item With a few country cases, illustrating the diversity of objectives and operational frameworks    
    \end{itemize}

  \item New risk-based IMF/MCM framework to time FX intervention rules from Lafarguette and Veyrune (2021) \href{https://www.elibrary.imf.org/view/journals/001/2021/032/001.2021.issue-032-en.xml}{\beamergotobutton{link}}
    \begin{itemize}
    \item Using the new Python package developped by Lafarguette and Raboun (2023) \href{https://pypi.org/project/varfxi/}{\beamergotobutton{link}}
    \end{itemize}

  \item Training on volatility modeling and the econometrics of time series  
    
  \item Training on programming with Python
        
  \end{wideenumerate}
\end{frame}



\begin{frame}
\frametitle{Objectives}
\begin{wideitemize}
  \item Our objective is to be \textbf{as helpful as possible}
  \item Very pragmatic, hands-on approach to use quantitative tools for central banks operations
    \begin{itemize}
    \item It goes beyond only FX interventions 
    \item Can be used for market monitoring, modeling, forecasting, etc.
    \item Build good foundations for your central bank and your career
    \item See our other STI course on liquidity forecasting
    \end{itemize}
  \item The tools we use are at the \textbf{best industry standards among quants}
    \begin{itemize}
    \item \textbf{Python}, the leading \textbf{free and open-source} programming language for data science and modelization (among other uses)
    \item \textbf{Jupyter notebooks and book} for full reproducibilty and easy deployment
    \item We \textbf{deploy our package on pypi}, for easy maintenance and update
    \item \textbf{Course website} deployed with Github workflow: \href{https://github.com/amineraboun/STI_FX_Intervention}{\beamergotobutton{Link}}
    \end{itemize}
\end{wideitemize}
\end{frame}

\begin{frame}
  \frametitle{The Team}
  \begin{wideitemize}
  \item \textbf{Romain Lafarguette, Ph.D.}: currently buy-side quant at ADIA (the sovereign wealth fund of Abu Dhabi) Q team
    \begin{itemize}
    \item Former IMF expert and ECB market operations expert (10 years)
    \item Missions to more than 25 countries, including as mission chief, modeling publications, teaching, etc.
    \item Mission coverage includes China PBoC, Reserve Bank of India, Hong Kong Monetary Authorities, Israel, Peru, Morocco, Tunisia, Algeria, Bosnia, WAEMU, etc.  
    \end{itemize}
    
  \item \textbf{Amine Rabound, Ph.D.}: buy-side quant at ADIA Q team
    \begin{itemize}
    \item Specialized on financial modeling and market microstructure
    \item Author of articles and a \href{https://worldscientific.com/worldscibooks/10.1142/12731}{\beamergotobutton{book}} on financial economics and econometrics
    \end{itemize}
    
  \item \textbf{Zhuohui Chen}: research analyst at the IMF, Monetary and Capital Markets department, central banks operation division.
    \begin{itemize}
    \item In charge of deploying quantitative tools to central banks, a dozen of countries missions (and more to come !)
    \end{itemize}
    
  \end{wideitemize}
  
\end{frame}


\begin{frame}
  \frametitle{Organization}

  \begin{wideitemize}
    \item We have opted for a blend between \textbf{plenary sessions} and \textbf{workshops}
    \item The workshops will provide the opportunity to gain and apply modeling and programming knowledge
    \item The idea is really to help participants developing useful, ready-to-use skills for their tasks at their central bank
    \item \textbf{Don't hesitate to ask questions}, \textbf{we are here to help}
      \begin{itemize}
      \item We will adjust the pace of the course depending on participants, please let us know
      \end{itemize}
  \end{wideitemize}
  
\end{frame}


\begin{frame}
  \frametitle{Program}
  \begin{wideitemize}
    \item \textbf{Monday}: Theory and country cases on FX interventions (morning), introduction to Python (afternoon, with workshop)
    \item \textbf{Tuesday}: Fundamentals of statistics (theory and workshop), fundamentals of time-series econometrics (theory and workshop)
    \item \textbf{Wednesday}: Fundamentals of volatility modeling (theory and workshop), forecasting evaluation 
    \item \textbf{Tuesday}: Risk-based framework for FX interventions: theory and workshop with test on real countries data
    \item \textbf{Friday}: Participants deployment of the codes, interpreation and reporting on their own country data
  \end{wideitemize}
  
\end{frame}



%% ---------------------------------------------------------------------------
%% End document
%% ---------------------------------------------------------------------------
\end{document}


%%% Local Variables:
%%% mode: latex
%%% TeX-master: t
%%% End:
