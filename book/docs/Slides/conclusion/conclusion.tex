%% ---------------------------------------------------------------------------
%% Preamble: Packages and Setup
%% ---------------------------------------------------------------------------
% Class 
\documentclass{beamer}

% Theme
\usetheme{Boadilla}
\usecolortheme{dolphin}
%\setbeamertemplate{headline}{} % Remove the top navigation bar

% Font and encoding
\usepackage[utf8]{inputenc} % Input font
\usepackage[T1]{fontenc} % Output font
\usepackage{lmodern} % Standard LateX font
\usefonttheme{serif} % Standard LateX font

% Maths 
\usepackage{amsfonts, amsmath, mathabx, bm, bbm} % Maths Fonts

% Graphics
\usepackage{graphicx} % Insert graphics
\usepackage{subfig} % Multiple figures in one graphic
\graphicspath{{/img}}

% Layout
\usepackage{changepage}

% Colors
\usepackage{xcolor}
\definecolor{imfblue}{RGB}{0,76,151} % Official IMF color
\setbeamercolor{title}{fg=imfblue}
\setbeamercolor{frametitle}{fg=imfblue}
\setbeamercolor{structure}{fg=imfblue}

% Tables
\usepackage{booktabs,rotating,multirow} % Tabular rules and other macros
%\usepackage{pdflscape,afterpage} % Landscape mode and afterpage
%\usepackage{threeparttable} % Split long tables
\usepackage[font=scriptsize,labelfont=scriptsize,labelfont={color=imfblue}]{caption}

% Import files
\usepackage{import}

% Appendix slides
\usepackage{appendixnumberbeamer} % Manage page numbers for appendix slides

% References
\usepackage{hyperref}

% A few macros: environments
\newenvironment{wideitemize}{\itemize\addtolength{\itemsep}{10pt}}{\enditemize}
\newenvironment{wideenumerate}{\enumerate\addtolength{\itemsep}{10pt}}{\endenumerate}

\newenvironment{extrawideitemize}{\itemize\addtolength{\itemsep}{30pt}}{\enditemize}
\newenvironment{extrawideenumerate}{\enumerate\addtolength{\itemsep}{30pt}}{\endenumerate}

% Remove navigation symbols and other superfluous elements
\setbeamertemplate{navigation symbols}{}
\beamertemplatenavigationsymbolsempty

%\setbeamertemplate{note page}[plain]
\hypersetup{pdfpagemode=UseNone} % don't show bookmarks on initial view
\setbeameroption{hide notes}

% Institute font
\setbeamerfont{institute}{size=\footnotesize}
\DeclareMathSizes{10}{9}{7}{5}

% Footnote without marker
\newcommand\blfootnote[1]{%
  \begingroup
  \renewcommand\thefootnote{}\footnote{#1}%
  \addtocounter{footnote}{-1}%
  \endgroup
}

%% ---------------------------------------------------------------------------
%% Title info
%% ---------------------------------------------------------------------------
\title[Key Points]{Key Points to Remember \\ STI Course on FX Interventions}

\author[Lafarguette]{Romain Lafarguette \and Zhuohui Chen  \and Amine Raboun}
\institute[IMF STX]{Quants \& IMF MCM \& IMF External Experts\blfootnote{\scriptsize{\emph{This training material is the property of the IMF, any reuse requires IMF permission}}} \\
  % \begin{center}{\href{https://romainlafarguette.github.io/}{\textcolor{imfblue}{romainlafarguette.github.io/}} \hspace{0.3cm} \href{https://amineraboun.github.io/}{\textcolor{imfblue}{amineraboun.github.io/}}} \end{center} \vspace{-0.5cm}
} 

\date[STI, 21 April 2023]{\footnotesize Singapore Training Institute, 21 April 2023}

\titlegraphic{\vspace{-0.6cm}
    \begin{figure}
    \centering
    \subfloat{{\includegraphics[width=2cm]{img/imf_logo}}}%
    \end{figure}}

% Slide between sections
\AtBeginSection[]
{
    \begin{frame}
        \tableofcontents[currentsection]
    \end{frame}
}

%% ---------------------------------------------------------------------------
%% Title slide
%% ---------------------------------------------------------------------------
\begin{document}

\begin{frame}
\maketitle
\end{frame}



%% ---------------------------------------------------------------------------
%% Remember
%% ---------------------------------------------------------------------------
\begin{frame}
  \frametitle{FX Interventions}
  \begin{wideitemize}
  \item FX reserves come \textbf{with benefits and costs}
    \begin{itemize}
    \item Buffer for intervention, building credibility, reducing risk premia, countercyclical policy
    \item Cost of sterilization (liquidity management), valuation/duration risk on the balance sheet
    \end{itemize}
  \item Choose the \textbf{right instrument} depending on the issue to address:
    \begin{itemize}
    \item Excessive volatility: FX spot or forward
    \item FX speculation witout endangering FX reserves: \textbf{non-deliverable forward}
    \item Funding issue without risk transfer: \textbf{FX swaps}
    \end{itemize}
  \item \textbf{Implementation}: pre-announced auctions \textbf{maximize signalling}, \textbf{reduce market distortions}
  \item Can operate rule-based interventions and discretionary interventions if necessary
  \end{wideitemize}

  
\end{frame}


\begin{frame}
  \frametitle{Risk-Based FX Interventions}
  \begin{wideitemize}
   \item For floating currencies, FX Interventions are about \textbf{managing risk at the macro level}
   \item \textbf{Central banks should model and anticipate risk}, both for discretionary and rules-based interventions
    \item Intervening through risk-based rules preserves exchange rate as \textbf{shock absorber} while strengthening financial stability
    \item The risk model can be used to benchmark FX interventions, even discretionary ones
    \item The VaR FXI spot rule can be \textbf{combined with other types of interventions}, for instance auctions on forward and NDF
    \item The risk tolerance can be dynamically ajusted as the hedging instruments are becoming more available  
  \end{wideitemize}
\end{frame}


\begin{frame}
  \frametitle{Statistics}
  \begin{wideitemize}
    \item The mean, mode and median inform about the \textbf{central tendency}
    \item The variance informs about the \textbf{volatility/risk}, and the skewness about the \textsc{balance of risks} (asymmetry)
    \item \textbf{Density forecasts} are important for policymakers: think about risks and balance of risks, not just the central tendency
    \item Density forecasts: \textbf{uncertainty about the world}
    \item Confidence interval of the point forecasts: \textbf{uncertainty about the model}
  \end{wideitemize}

\end{frame}


\begin{frame}
  \frametitle{Time Series Models}
  \begin{wideitemize}
  \item Always check for \textbf{stationarity} before estimating models (ADF, KPSS): \textbf{differentiate} the data as necessary
  \item \textbf{Seasonality} and \textbf{trend} are generating non-stationarity
    \begin{itemize}
    \item Seasonality can be dealt with Fourier terms (if regular), dummies (if irregular)
    \item Trend can be addressed with detrending methods
    \end{itemize}
  \item Partial autocorrelation purges out the effect of terms at shorter lags
  \item PACF: help selecting lag from AR, ACF helps on MA models
  \item Check that the residuals look like a white noise, for well-specified model
  \end{wideitemize}
\end{frame}


\begin{frame}
  \frametitle{Volatility Modeling}
  \begin{wideitemize}
    \item There is a difference between \textbf{realized volatility} (of realizations over time) and \textbf{conditional volatility} (probability of a future realization)
    \item GARCH models address most of the features of financial time series, yet are relatively simple too
    \item Very few models are able to outperform GARCH(1, 1). \textbf{GARCH(1,1) should be the benchmark model}
    \item GARCH embedds symmetric volatility shock. If you want \textbf{asymmetric shocks}: use \textbf{GJR-GARCH or EGARCH} instead.
    \item If you need more skewness and kurtosis to fit the data, change the distribution of the normalized residuals: SkewStudent, GED, etc.
  \end{wideitemize}
\end{frame}



\begin{frame}
  \frametitle{Model Validation}
 \begin{wideitemize}
    \item A model that performs well in-sample (R2, significant coefficients, etc.) does NOT necessarily forecast well: \textbf{risk of overfit}
  \item Crucial to measure the \textbf{out-of-sample performance}: backtest in \textbf{"real-conditions"}
  \item Use the right metric depending on the problem at hand:
    \begin{itemize}
    \item For point forecast: RMSE, MSE, MAE (without outliers), MAPE. RMSE and MSE are sensitive to outliers
    \item For density forecasts:
      \begin{itemize}
      \item Specification test: \textbf{Probability integral transform} (not too optimistic/pessimistic)
      \item Performance test: \textbf{logscore} (on average) or asymmetric logscore (for the tails)
      \end{itemize}
    \end{itemize}
   \item Hypothesis testing: "\emph{Can not reject H0 doesn't mean we can accept HA}"
\end{wideitemize}
\end{frame}


\begin{frame}
  \frametitle{Python Programming}
  \begin{wideitemize}
  \item Python is free and open-source and \textbf{can help the central bank for many tasks}, with thousands of packages:
    \begin{itemize}
    \item Modeling and forecasting
    \item Data management and data vizualization
    \item Tasks automation
    \end{itemize}
    \item You can use \textbf{Jupyter Notebooks} for coding, simple and efficient. Else Spyder for more advanced tools
    \item Create a new environment for each new project via Anaconda
    \item Import packages before using them, and set the working directory before running script
    \item \textbf{Lot of free online resources to learn Python}: encourage colleagues to use Python!
  \end{wideitemize}

\end{frame}




\begin{frame}
  \frametitle{Thank You}
  \begin{wideitemize}
  \item \textbf{Thank you very much for your enthusiasm and participation!}
  \item Thanks to STI and Alina for the spotless organization!
  \item We hope that this course will be helpful to you
    \begin{itemize}
    \item Not only for FX interventions
    \item Also for monitoring, research, forecasting, etc.
    \end{itemize}
  \item Quantitative methods are more and more important, important for your central bank and your career !
  \item Interested in quantitative methods on other topics?
    \begin{itemize}
    \item We are teaching \textbf{Forecasting Framework for Central Bank Systemic Liquidity in STI in 27 November - 01 December 2023 !}
    \item Don't hesitate to register! FFCBSL: \href{https://www-ins.imf.org/TAS/signon.aspx?pkey=ST23.30E}{https://www-ins.imf.org/TAS/signon.aspx?pkey=ST23.30E}
    \end{itemize}
  \end{wideitemize}
\end{frame}




%% ---------------------------------------------------------------------------
%% End document
%% ---------------------------------------------------------------------------
\end{document}

%%% Local Variables:
%%% mode: latex
%%% TeX-master: t
%%% End:
